\documentclass{article}
\usepackage{fullpage,amssymb,amsmath,epsf}
\usepackage[12pt]{extsizes}
\usepackage{mathtools}
\usepackage{graphicx}
\usepackage{epstopdf}
\numberwithin{equation}{section}

\setlength{\textheight}{8in}

\usepackage[hang,flushmargin]{footmisc}

\usepackage{url}

\newcommand\blfootnote[1]{%
	\begingroup
	\renewcommand\thefootnote{}\footnote{#1}%
	\addtocounter{footnote}{-1}%
	\endgroup
}

\pagestyle{myheadings}

\newcommand{\newsec}{\section}
\newcommand{\denselist}{\itemsep 0pt\partopsep 0pt}
\newcommand{\bitem}{\begin{itemize}\denselist}
\newcommand{\eitem}{\end{itemize}}
\newcommand{\benum}{\begin{enumerate}\denselist}
\newcommand{\eenum}{\end{enumerate}}

\newcommand{\fig}[1]{\private{\begin{center}
{\Large\bf ({#1})}
\end{center}}}

\newcommand{\cpsf}[1]{{\centerline{\psfig{#1}}}}
\newcommand{\mytitle}[1]{\centerline{\LARGE\bf #1}}

\newcommand{\myw}{{\bf w}}

\newcommand{\mypar}[1]{\vspace{1ex}\noindent{\bf {#1}}}

\def\thmcolon{\hspace{-.85em} {\bf :} }

\newtheorem{THEOREM}{Theorem}[section]
\newenvironment{theorem}{\begin{THEOREM} \thmcolon }%
                        {\end{THEOREM}}
\newtheorem{LEMMA}[THEOREM]{Lemma}
\newenvironment{lemma}{\begin{LEMMA} \thmcolon }%
                      {\end{LEMMA}}
\newtheorem{COROLLARY}[THEOREM]{Corollary}
\newenvironment{corollary}{\begin{COROLLARY} \thmcolon }%
                          {\end{COROLLARY}}
\newtheorem{PROPOSITION}[THEOREM]{Proposition}
\newenvironment{proposition}{\begin{PROPOSITION} \thmcolon }%
                            {\end{PROPOSITION}}
\newtheorem{DEFINITION}[THEOREM]{Definition}
\newenvironment{definition}{\begin{DEFINITION} \thmcolon \rm}%
                            {\end{DEFINITION}}
\newtheorem{CLAIM}[THEOREM]{Claim}
\newenvironment{claim}{\begin{CLAIM} \thmcolon \rm}%
                            {\end{CLAIM}}
\newtheorem{EXAMPLE}[THEOREM]{Example}
\newenvironment{example}{\begin{EXAMPLE} \thmcolon \rm}%
                            {\end{EXAMPLE}}
\newtheorem{REMARK}[THEOREM]{Remark}
\newenvironment{remark}{\begin{REMARK} \thmcolon \rm}%
                            {\end{REMARK}}
%\newenvironment{proof}{\noindent {\bf Proof:} \hspace{.677em}}%
%                      {}

%theorem
\newcommand{\thm}{\begin{theorem}}
%lemma
\newcommand{\lem}{\begin{lemma}}
%proposition
\newcommand{\pro}{\begin{proposition}}
%definition
\newcommand{\dfn}{\begin{definition}}
%remark
\newcommand{\rem}{\begin{remark}}
%example
\newcommand{\xam}{\begin{example}}
%corollary
\newcommand{\cor}{\begin{corollary}}
%proof
\newcommand{\prf}{\noindent{\bf Proof:} }
%end theorem
\newcommand{\ethm}{\end{theorem}}
%end lemma
\newcommand{\elem}{\end{lemma}}
%end proposition
\newcommand{\epro}{\end{proposition}}
%end definition
\newcommand{\edfn}{\bbox\end{definition}}
%end remark
\newcommand{\erem}{\bbox\end{remark}}
%end example
\newcommand{\exam}{\bbox\end{example}}
%end corollary
\newcommand{\ecor}{\end{corollary}}
%end proof
\newcommand{\eprf}{\bbox\vspace{0.1in}}
%begin equation
\newcommand{\beqn}{\begin{equation}}
%end equation
\newcommand{\eeqn}{\end{equation}}

%\newcommand{\eqref}[1]{Eq.~\ref{#1}}

\newcommand{\KB}{\mbox{\it KB\/}}
\newcommand{\infers}{\vdash}
\newcommand{\sat}{\models}
\newcommand{\bbox}{\vrule height7pt width4pt depth1pt}

\newcommand{\act}[1]{\stackrel{{#1}}{\rightarrow}}
\newcommand{\at}[1]{^{(#1)}}

\newcommand{\argmax}{{\rm argmax}}

\newcommand{\rimp}{\Rightarrow}
\newcommand{\dimp}{\Leftrightarrow}

\newcommand{\bX}{\mbox{\boldmath $X$}}
\newcommand{\bY}{\mbox{\boldmath $Y$}}
\newcommand{\bZ}{\mbox{\boldmath $Z$}}
\newcommand{\bU}{\mbox{\boldmath $U$}}
\newcommand{\bE}{\mbox{\boldmath $E$}}
\newcommand{\bx}{\mbox{\boldmath $x$}}
\newcommand{\be}{\mbox{\boldmath $e$}}
\newcommand{\by}{\mbox{\boldmath $y$}}
\newcommand{\bz}{\mbox{\boldmath $z$}}
\newcommand{\bu}{\mbox{\boldmath $u$}}
\newcommand{\bd}{\mbox{\boldmath $d$}}
\newcommand{\smbx}{\mbox{\boldmath $\scriptstyle x$}}
\newcommand{\smbd}{\mbox{\boldmath $\scriptstyle d$}}
\newcommand{\smby}{\mbox{\boldmath $\scriptstyle y$}}
\newcommand{\smbe}{\mbox{\boldmath $\scriptstyle e$}}


\newcommand{\word}[1]{\mbox{\it #1\/}}
\newcommand{\Action}{\word{Action}}
\newcommand{\Proposition}{\word{Proposition}}
\newcommand{\true}{\word{true}}
\newcommand{\false}{\word{false}}
\newcommand{\Pre}{\word{Pre}}
\newcommand{\Add}{\word{Add}}
\newcommand{\Del}{\word{Del}}
\newcommand{\Result}{\word{Result}}
\newcommand{\Regress}{\word{Regress}}
\newcommand{\Maintain}{\word{Maintain}}

\newcommand{\bor}{\bigvee}
\newcommand{\invert}[1]{{#1}^{-1}}

\newcommand{\commentout}[1]{}

\newcommand{\bmu}{\mbox{\boldmath $\mu$}}
\newcommand{\btheta}{\mbox{\boldmath $\theta$}}
\newcommand{\IR}{\mbox{$I\!\!R$}}

\newcommand{\tval}[1]{{#1}^{1}}
\newcommand{\fval}[1]{{#1}^{0}}

\newcommand{\tr}{{\rm tr}}
\newcommand{\vecy}{{\vec{y}}}
\renewcommand{\Re}{{\mathbb R}}

\def\twofigbox#1#2{%
\noindent\begin{minipage}{\textwidth}%
\epsfxsize=0.35\maxfigwidth
\noindent \epsffile{#1}\hfill
\epsfxsize=0.35\maxfigwidth
\epsffile{#2}\\
\makebox[0.35\textwidth]{(a)}\hfill\makebox[0.35\textwidth]{(b)}%
\end{minipage}}

\def\twofigboxcd#1#2{%
\noindent\begin{minipage}{\textwidth}%
\epsfxsize=0.35\maxfigwidth
\noindent \epsffile{#1}\hfill
\epsfxsize=0.35\maxfigwidth
\epsffile{#2}\\
\makebox[0.35\textwidth]{(c)}\hfill\makebox[0.35\textwidth]{(d)}%
\end{minipage}}

\def\twofigboxnolabel#1#2{%
\begin{minipage}{\textwidth}%
\epsfxsize=0.35\maxfigwidth
\noindent \epsffile{#1}\hfill
\epsfxsize=0.35\maxfigwidth
\epsffile{#2}\\
%\makebox[0.48\textwidth]{(a)}\hfill\makebox[0.48\textwidth]{(b)}%
\end{minipage}
}

\def\twofigboxnolabelFive#1#2{%
\begin{minipage}{\textwidth}%
\hbox to 0.5in{}\epsfxsize=0.35\maxfigwidth
\noindent \epsffile{#1}\hfill
\epsfxsize=0.35\maxfigwidth
\epsffile{#2}\hbox to 0.5in{}\\
%\makebox[0.48\textwidth]{(a)}\hfill\makebox[0.48\textwidth]{(b)}%
\end{minipage}
}

\def\threefigbox#1#2#3{%
\noindent\begin{minipage}{\textwidth}%
\epsfxsize=0.33\maxfigwidth
\noindent \epsffile{#1}\hfill
\epsfxsize=0.33\maxfigwidth
\noindent \epsffile{#2}\hfill 
\epsfxsize=0.33\maxfigwidth
\epsffile{#3}\\
\makebox[0.31\textwidth]{{\scriptsize (a)}}\hfill%
\makebox[0.31\textwidth]{{\scriptsize (b)}}\hfill
\makebox[0.31\textwidth]{{\scriptsize (c)}}%
\smallskip
\end{minipage}}

\def\threefigboxnolabel#1#2#3{%
\noindent\begin{minipage}{\textwidth}%
\epsfxsize=0.33\maxfigwidth
\noindent \epsffile{#1}\hfill
\epsfxsize=0.33\maxfigwidth
\noindent \epsffile{#2}\hfill 
\epsfxsize=0.33\maxfigwidth
\epsffile{#3}\\
%\makebox[0.31\textwidth]{{\scriptsize (a)}}\hfill%
%\makebox[0.31\textwidth]{{\scriptsize (b)}}\hfill
%\makebox[0.31\textwidth]{{\scriptsize (c)}}%
%\smallskip
\end{minipage}}

\newlength{\maxfigwidth}
\setlength{\maxfigwidth}{\textwidth}
%\def\captionsize {\footnotesize}
\def\captionsize {}

\newcommand{\xsi}{{x^{(i)}}}
\newcommand{\xsd}{{x^{(d)}}}
\newcommand{\xsj}{{x^{(j)}}}
\newcommand{\ysi}{{y^{(i)}}}
\newcommand{\ysj}{{y^{(j)}}}
\newcommand{\gsi}{{\gamma^{(i)}}}
\newcommand{\wsi}{{w^{(i)}}}
\newcommand{\esi}{{\epsilon^{(i)}}}

\newcommand{\calM}{{\cal M}}
\newcommand{\calH}{{\cal H}}
\newcommand{\calN}{{\cal N}}
\newcommand{\calX}{{\cal X}}
\newcommand{\calY}{{\cal Y}}
\newcommand{\calL}{{\cal L}}
\newcommand{\calP}{{\cal P}}
\newcommand{\calD}{{\cal D}}
\newcommand{\calF}{{\cal F}}

\newcommand{\ytil}{{\tilde{y}}}

\newcommand{\Ber}{{\rm Bernoulli}}
\newcommand{\MI}{{\rm MI}}
\newcommand{\E}{{\rm E}}

\newcommand{\pstar}{{p^{\ast}}}
\newcommand{\bstar}{{b^{\ast}}}
\newcommand{\dstar}{{d^{\ast}}}
\newcommand{\wstar}{{w^{\ast}}}
\newcommand{\alphastar}{\alpha^{\ast}}
\newcommand{\alphastari}{{\alpha_i^{\ast}}}
\newcommand{\betastar}{{\beta^{\ast}}}
\newcommand{\tol}{{\textit tol}}
\newcommand{\phihat}{\hat\phi}
\newcommand{\ehat}{\hat\varepsilon}
\newcommand{\hhat}{\hat{h}}
\newcommand{\hstar}{h^\ast}
\newcommand{\VC}{{\rm VC}}

\newcommand{\Strain}{S_{\rm train}}
\newcommand{\Scv}{{S_{\rm cv}}}

\newcommand{\hwb}{{h_{w,b}}}

\usepackage{graphicx}
\usepackage{subcaption}
%\renewcommand{\epsffile}[1]{
%	\includegraphics[width=\epsfxsize]{#1}
%}

\newcommand{\di}{{d}}
\newcommand{\nexp}{{n}}
\newcommand{\vcd}{{\textbf{D}}}


\usepackage{titling}
\setlength{\droptitle}{-5em}


\begin{document}
\title{XCS229i: Additional Notes on Backpropagation}
\date{}
\maketitle

\section{Forward propagation}

Recall that given input $x$, we define $a^{[0]} = x$.
Then for layer $\ell = 1,2,\dots,N$, where $N$ is the number of layers of
the network, we have
\begin{enumerate}
  \item $z^{[\ell]} = W^{[\ell]} a^{[\ell-1]} + b^{[\ell]}$
  \item $a^{[\ell]} = g^{[\ell]}(z^{[\ell]})$
\end{enumerate}
In these notes we assume the nonlinearities $g^{[\ell]}$ are the same for all layers
besides layer~$N$. This is because in the output layer we may be doing
regression [hence we might use $g(x) = x$] or binary classification
[$g(x) = \text{sigmoid}(x)$] or multiclass classification [$g(x) = \text{softmax}(x)$].
Hence we distinguish $g^{[N]}$ from $g$, and assume $g$ is used for all layers besides layer $N$.

Finally, given the output of the network $a^{[N]}$, which we will more simply denote as
$\hat{y}$, we measure the loss
$J(W, b) = \mathcal{L}(a^{[N]}, y) = \mathcal{L}(\hat{y}, y)$. For example, for real-valued regression we might use
the squared loss
$$\mathcal{L}(\hat{y}, y) = \frac{1}{2}(\hat{y} - y)^2$$
and for binary classification using logistic regression we use
$$\mathcal{L}(\hat{y}, y) = -(y \log \hat{y} + (1-y) \log(1-\hat{y}))$$
or negative log-likelihood. Finally, for softmax regression over $k$ classes, we use
the cross entropy loss
$$\mathcal{L}(\hat{y}, y) = -\sum_{j=1}^k \mathbf{1}\{y=j\} \log \hat{y}_j$$
which is simply negative log-likelihood extended to the multiclass setting.
Note that $\hat{y}$ is a $k$-dimensional vector in this case.
If we use $y$ to instead denote the $k$-dimensional vector of zeros with
a single $1$ at the $l$th position, where the true label is $l$, we can also
express the cross entropy loss as
$$\mathcal{L}(\hat{y}, y) = -\sum_{j=1}^k y_j \log \hat{y}_j$$

\section{Backpropagation}

Let's define one more piece of notation that'll be useful for
backpropagation.\footnote{These notes are closely adapted from:\\\url{http://ufldl.stanford.edu/tutorial/supervised/MultiLayerNeuralNetworks/}\\Scribe: Ziang Xie}
We will define
$$\delta^{[\ell]} = \nabla_{z^{[\ell]}} \mathcal{L}(\hat{y}, y)$$

We can then define a three-step  ``recipe" for computing the gradients with respect to
every $W^{[\ell]}, b^{[\ell]}$ as follows:
\begin{enumerate}
  \item For output layer $N$, we have
    $$\delta^{[N]} = \nabla_{z^{[N]}} \mathcal{L}(\hat{y}, y)$$
    Sometimes we may want to compute $\nabla_{z^{[N]}} \mathcal{L}(\hat{y}, y)$ directly
    (e.g. if $g^{[N]}$ is the softmax function),
    whereas other times (e.g. when $g^{[N]}$ is the sigmoid function $\sigma$)
    we can apply the chain rule:
    $$\nabla_{z^{[N]}} \mathcal{L}(\hat{y}, y) = \nabla_{\hat{y}} \mathcal{L}(\hat{y}, y) \circ (g^{[N]})'(z^{[N]})$$
    Note $(g^{[N]})'(z^{[N]})$ denotes the elementwise derivative w.r.t. $z^{[N]}$.
  \item For $\ell = N-1, N-2, \dots, 1$, we have
    $$\delta^{[\ell]} = ({W^{[\ell+1]}}^\top \delta^{[\ell + 1]}) \circ g'(z^{[\ell]})$$
  \item Finally, we can compute the gradients for layer $\ell$ as
    \begin{align*}
      \nabla_{W^{[\ell]}} J(W, b) &= \delta^{[\ell]} {a^{[\ell-1]}}^\top\\
      \nabla_{b^{[\ell]}} J(W, b) &= \delta^{[\ell]}
    \end{align*}
\end{enumerate}
where we use $\circ$ to indicate the elementwise product.
Note the above procedure is for a single training example.

You can try applying the above algorithm to logistic regression
($N=1$, $g^{[1]}$ is the sigmoid function $\sigma$) to sanity check steps (1) and (3).
Recall that $\sigma'(z) = \sigma(z) \circ (1-\sigma(z))$ and $\sigma(z^{[1]})$ is simply $a^{[1]}$.
Note that for logistic regression, if $x$ is a column vector in $\mathbb{R}^{\di\times 1}$, then
$W^{[1]} \in \mathbb{R}^{1 \times \di}$, and hence $\nabla_{W^{[1]}} J(W, b) \in \mathbb{R}^{1 \times \di}$.
Example code for two layers is also given at:
\begin{center}
\url{http://cs229.stanford.edu/notes/backprop.py}
\end{center}

% TODO Vectorization.

\end{document}
